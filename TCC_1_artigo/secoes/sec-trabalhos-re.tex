\section{Trabalhos Relacionados}
\label{sec:trabalhos-re}

\todo[inline]{Nesta parte faz-se a revisão de literatura sobre o assunto, resumindo-se os resultados de estudos feitos por outros autores, cujas obras citadas e consultadas devem constar nas referências.}

\textcolor{red}{A função desta seção é demonstrar o conhecimento já existente na ciência. O que faltar de conhecimento é lacuna e, portanto, novidade que você pode explorar, tomando, então, como seu problema de pesquisa. Deve conter, no mínimo, três pesquisas em formato de artigo científico, preferencialmente, dos últimos cinco anos. Você poderá obter esses estudos por indicação do orientador ou, por exemplo, no Google Acadêmico.}

\textcolor{red}{As pesquisas que você irá buscar para compor esta seção são aquelas semelhantes à sua, ou seja, que investigam o mesmo tema e que atacam problemas de pesquisa nos quais você está interessado em estudar.}

\textcolor{red}{A descrição de um estudo deve ocupar, idealmente, apenas um parágrafo. Como regra geral, um parágrafo pode ter até 20 linhas, mas tente trabalhar com o máximo de 12 ou 16. No máximo, a sua descrição de um estudo deve ocupar dois parágrafos, não mais que isso.}

\textcolor{red}{As descrições de estudos devem ser conectadas entre si por meio do texto. Ou seja, você pode afirmar que Santos (2022) investigou X e conclui Y, sugerindo como trabalho futuro Z. Aí, no parágrafo seguinte, você pode dizer que Silva (2022), baseado em Santos (2022), conduziu estudo para investigar Z. Note, portanto, que os estudos foram, de algum modo, conectados no seu texto.}

\textcolor{red}{Como regra geral, a descrição de um estudo deve seguir o seguinte formato, que você pode adaptar: “Silva e Santos (2022) conduziram estudo cujo objetivo foi {Preencher com o verbo no infinitivo mais complemento adotado por Silva e Santos}. Para atingir esse objetivo, adotaram o seguinte método: {Preencher com uma breve descrição do método do estudo}. Os pesquisadores verificaram que {Preencher com os principais dados}. A partir desses dados, concluíram que {Preencher com a resposta à principal pergunta de pesquisa}. Esse estudo teve como principais limitações: {Preencher com uma breve síntese}”.}

\textcolor{red}{O parágrafo final desta seção pode explicitar a correlação entre as pesquisas descritas e, então, explicitar de modo inequívoco a lacuna na literatura que justifica o problema de pesquisa que, na seção seguinte, será apresentado.}

\todo[inline]{Deve-se apresentar uma visão geral e comparativa dos trabalhos apresentados, bem como, cada trabalho apresentado pode contribuir com o seu trabalho. Sugere-se também apresentar uma tabela comparativa entre os trabalhos apresentados. Abaixo segue um exemplo:}

A seguir analisamos as técnicas (ver Tabela~\ref{Table:TechByPapers}) 
adotadas para a geração das invariantes de programa nos trabalhos apresentados anteriormente. Vale ressaltar que as técnicas identificadas foram:

\begin{itemize}
    \item \textbf{Análise de Predicados}. Analisando os métodos baseadas na análise de predicado, podemos observar que a análise de predicado fornece um apoio significativo na análise do programa para inferir propriedades sobre o comportamento do programa;
    %
    \item \textbf{Interpretação Abstratata}. É uma teoria da aproximação da semântica de linguagens de programação cuja aplicação principal é a análise estática;
    %
    \item \textbf{Lógica de Mill}. A lógica de Mill é adotada para caracterizar laços por meio de uma função que define o seu espaço de estados.
    %
    \item \textbf{\textit{Templates}}. Analisando as publicações, notamos que a adoção de \textit{templates} é importante para fornecer um suporte para guiar a inferência de invariantes de programas.
\end{itemize}{}

%\multicolumn{number cols}{align}{text} % align: l,c,r
%\multirow{number rows}{width}{text}
\begin{table}[htbp]  
  \centering  
  \scalefont{0.9}  
  \begin{tabular}{|c|c|c|c|c|}
    \hline
    % header
    % row 1 and 2
    \multicolumn{1}{|c|}{\multirow{2}{*}{\textbf{Artigos}}} & 
    % row 1
    \multicolumn{4}{c|}{\textbf{Técnicas}} \bigstrut\\
    \cline{2-5} &
    % row 2
    \textbf{Predicados} & \textbf{Int. Abstrata} & \textbf{Log. Mill} & \textbf{\textit{Templates}} \bigstrut\\
    \hline
    % body table - row 3
    Trabalho 1 & X &  &  &  \bigstrut\\
    \hline
    % body table - row 4
    Trabalho 2 &  & X & X &  \bigstrut\\
    \hline
    % body table - row 5
    Trabalho 3 & X & X &  & X \bigstrut\\
    \hline
    % body table - row 6
    Trabalho 4 & X & X &  & X \bigstrut\\
    \hline
  \end{tabular}  
  \caption{Classificação dos artigos por técnicas.}
  \label{Table:TechByPapers}
\end{table}%
