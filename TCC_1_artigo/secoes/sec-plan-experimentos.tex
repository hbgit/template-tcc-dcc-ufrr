\section{Planejamento para Avaliação Experimental}
\label{sec:plan-experimentos}

\todo[inline]{\textbf{NOTA}: Deve-se adicionar um texto para introduzir o capitulo, segue um \textbf{exemplo}}

Esta seção descreve o planejamento para a execução da avaliação experimental da solução proposta, incluindo: o planejamento e projeto para a execução de um estudo experimental para avaliar o método proposto 
neste trabalho.


\subsection{Projeto da Avaliação Experimental}
\label{sub:experimento}

\todo[inline]{\textbf{NOTA}: Deve-se descrever o projeto para executar os testes para avaliar a solução proposta, incluindo o ambiente, elementos que serão utilizados, artefatos de entrada e saída, dados que serão coletados, e métricas para avaliação. Segue um \textbf{exemplo} abaixo.}

\textcolor{red}{Esta seção descreve o planeamento e concepção para a execução de um estudo empírico realizado com o objetivo de avaliar a solução proposta para \textcolor{red}{XXXXX}. O estudo será conduzido aplicando a solução proposta sobre \textit{benchmarks} públicos de programas em C. Os experimentos foram conduzidos em um computador Intel Xeon CPU E5, 2.60GHz, 115GB RAM com Linux 3.13.0 - 35-generic x86\_64.}

\todo[inline]{\textbf{NOTA}: Deve-se também apresentar como será executado a avaliação e qual o seu objetivo de cada ação na avaliação. Neste momento você deve considerar as formas de testar (cenários importante) e como coletar os dados para sua avaliação. Segue um \textbf{exemplo}. } 

\textcolor{red}{Esta avaliação empírica tem como objetivo analisar a capacidade do método proposto, sobre benchmarks públicos de programas em C, para contribuir com a verificação executada pelo software $X$. Desta forma, nesta avaliação, investiga-se as seguintes questões de pesquisa (QP):
%
\begin{itemize}
    \item[QP1]: As ferramentas para a geração de invariantes são capazes de suportar as diferentes estruturas da linguagem de programação C?
    %
    \item[QP2]: As abordagens propostas para geração de dados de teste contribuem para a geração de invariantes?
    %
    \item[QP3]: As invariantes geradas contribuem na verificação executada pelo ESBMC?
\end{itemize}}


\todo[inline]{\textbf{NOTA}: Deve-se também apresentar também as etapas da experimentação.}
