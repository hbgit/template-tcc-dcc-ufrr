\section{Introdução}
\label{sec:introducao}

\todo[inline]{\textbf{NOTA}: O texto aqui apresentado é apenas a informação sobre algumas regras, para o escrita do seu trabalho, o texto abaixo DEVE SER REMOVIDO.}

\textcolor{red}{A função da introdução é informar ao leitor o tema da pesquisa, destacar a sua relevância científica e/ou social, sendo finalizada com a descrição de como o TCC está organizado, ou seja, como está subdividido. É importante ressaltar que a introdução apresenta a visão geral do TCC. Ela precisa ser sedutora, de forma a mobilizar o leitor para seguir com a leitura. Apesar de ser o primeiro capítulo do trabalho, normalmente é o último a ser feito. O motivo é que só apresentamos aquilo que já escrevemos.}

\textcolor{red}{1) Comece com uma sentença descrevendo qual é o tema da sua pesquisa, já podendo indicar o que você pretende investigar. 2) Na sequência, você pode dar uma ideia muito geral do que tem sido investigado nesse tema e já passar para a apresentação dos motivos para a realização da pesquisa. Alternativamente, pode já começar pela motivação. Motivos são argumentos, isto é, justificativas que indicam por qual motivo a pesquisa foi realizada. Argumente, preferencialmente, com base em dados, números, de pesquisa por qual motivo esse tema é relevante para a ciência e/ou para a sociedade. O ideal seria ter de dois a três argumentos para justificar a importância da sua pesquisa. Use citações para sustentar os argumentos. Os seus argumentos também podem mostrar a relevância do tema a partir das implicações que ele possui para a sociedade.}

\textcolor{red}{3) Você deve, então, explicitar o que irá estudar dentro do seu tema, o que significa explicitar o seu problema de pesquisa. O problema pode ser apresentado de duas formas: como objetivo (verbo no infinitivo + complemento) ou como pergunta. No caso do TCC do DCC, vamos mesclar as duas estratégias, apresentando o objetivo geral do estudo e perguntas de pesquisa que, se respondidas, atendem ao objetivo geral.
[4) Finalize a introdução com a descrição das seções que compõem o seu trabalho.}

\textcolor{red}{Atenção com relação à estrutura explicada. De modo simples, ela propõe que você responda três perguntas, sendo uma resposta por parágrafo: 1) Qual é o tema da pesquisa e o que você irá estudar dentro dessa temática? 2) Quais os motivos que justificam o que você pretende estudar? 3) Como o seu TCC está organizado?}

\todo[inline]{\textbf{NOTA}: Qual o problema que você está tentando resolver através do trabalho? Quais as restrições de projeto envolvidas?}

O problema considerado neste trabalho é expresso na seguinte questão:

\todo[inline]{A pergunta de pesquisa precisa ser respondida para que o objetivo geral possa ser alcançado. Pensando nisso, elabore a sua questão. O seu principal compromisso na pesquisa é conseguir essa resposta. Você pode ter uma só pergunta ou mais. Relembrando: você precisará responder essas perguntas, portanto é razoável não escrever qualquer questão.}

O objetivo deste estudo é 
\todo[inline]{Preencher com texto iniciando com um verbo no infinitivo seguido de complemento. Escolha um dos seguintes verbos e complementos: “Conduzir revisão de literatura sobre…”, “Caracterizar a variável…”, “Descrever a variável…”, “Desenvolver sistema/aplicativo/aplicação/script…”, “Investigar validade…”, “Testar validade do modelo…”, “Testar a relação de associação entre as variáveis A e B”, “Testar a relação de determinação entre as variáveis A e B” etc.}

\textbf{Organização do Trabalho.} As demais seções restantes são organizados da seguinte forma:
% 
Nos \textbf{Fundamentos Teóricos}, são apresentados os conceitos abordados neste trabalho, especificamente: textcolor{red}{XXX, YYY, e ZZZ.}
%
Nos \textbf{Trabalhos Correlatos}, são analisados os trabalhos correlatos a solução proposta.
%
Na \textbf{Método da Solução Proposta}, é descrito as etapas de execução do método da solução proposta para textcolor{red}{XXXX}.
%
No \textbf{Planejamento para Avaliação Experimental}, é apresentado o planejamento e projeto para execução da avaliação da solução proposta.
%
E por fim nas \textbf{Considerações Parciais}, apresenta-se as considerações parciais e análise das atividades já desenvolvidas.