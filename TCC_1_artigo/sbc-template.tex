\documentclass[12pt]{article}

\usepackage{estilo/sbc-template}
\usepackage{tikz}
\usepackage{transparent}
\usepackage{graphicx,url}
\usepackage[utf8]{inputenc}
\usepackage[brazil]{babel}
% para anotações no período de revisões
\usepackage{todonotes}
\usepackage{fancyhdr}
\usepackage{multirow}
\usepackage{scalefnt}
\usepackage{float}
\usepackage{datetime}
\usepackage{ifthen}


% --------------------------------------
% Definições de formatação
% Númeração das páginas
\pagestyle{fancy}
\fancyhf{}
\fancyhead[R]{\thepage}
\renewcommand{\headrulewidth}{0pt}


\sloppy

% --------------------------------------
% Definições de informações do TCC
% --------------------------------------
\title{TÍTULO DO TRABALHO}

\author{NOME DO DISCENTE} %\inst{1}

\orientador{Prof. Título. Nome Completo}
% Caso não tenha coorientador, deixar o campo vazio.
\coorientador{}
% Caso tenha coorientador, descomentar linha abaixo.
%\coorientador{Prof. Título. Nome Completo}

\address{
  Curso de Bacharelado em Ciência da Computação\\ 
  Departamento de Ciência da Computação\\ 
  Universidade Federal de Roraima (UFRR)\\ 
  Boa Vista -- RR -- Brasil  
  \email{\{aluno\}@ufrr.br}
}
% --------------------------------------


% --------------------------------------
% Início do documento
% --------------------------------------
\begin{document} 

% Não é necessário alterar
% Capa personalizada para UFRR
\begin{titlepage}
    \begin{tikzpicture}[remember picture,overlay]
        \node[opacity=0.1,inner sep=0pt] at (current page.center){
            \includegraphics[width=0.9\paperwidth]{estilo/dcc-logo.png}
        };
    \end{tikzpicture} 

    \begin{center}
        % Imagem no topo (troque o nome do arquivo para o correto)
        \includegraphics[width=0.3\textwidth]{estilo/ufrr-logo.png} \\[2cm]

        % Instituição
        {\large
        UNIVERSIDADE FEDERAL DE RORAIMA\\
        CENTRO DE CIÊNCIAS E TECNOLOGIA\\
        DEPARTAMENTO DE CIÊNCIA DA COMPUTAÇÃO\\
        CURSO DE BACHARELADO EM CIÊNCIA DA COMPUTAÇÃO\\[3cm]
        }

        % Nome do aluno
        {\bfseries\Large 
            \makeatletter
            \@author
            \makeatother
        }\\[1.5cm]

        % Título do trabalho
        {\bfseries\LARGE 
            \makeatletter
            \@title
            \makeatother
        }\\[3.5cm]

        % Data
        {\large Boa Vista - RR\\
        Maio de 2025}
    \end{center}
\end{titlepage}

% Folha de rosto
%\newpage
\begin{titlepage}
    \begin{tikzpicture}[remember picture,overlay]
        \node[opacity=0.1,inner sep=0pt] at (current page.center){
            \includegraphics[width=0.9\paperwidth]{estilo/dcc-logo.png}
        };
    \end{tikzpicture}
    \begin{center}
        % Título em negrito no topo
        {\bfseries\LARGE 
            \makeatletter
            \@title
            \makeatother
        }\\[4cm]

        % Nome do discente
        {\Large 
            \makeatletter
            \@author
            \makeatother
        }\\[4cm]

        % Texto com recuo à direita
        \begin{flushright}
            \parbox{9cm}{\setlength{\parindent}{0pt}
            Trabalho de Conclusão de Curso apresentado ao curso de Bacharelado em Ciência da Computação do Departamento de Ciência da Computação da Universidade Federal de Roraima, em cumprimento às exigências legais como requisito parcial à obtenção do título Bacharel em Ciência da Computação.
            }
        \end{flushright}

        \begin{flushright}
            \parbox{9cm}{\setlength{\parindent}{0pt}
            \textbf{Orientador:} \getorientador.
            }
        \end{flushright}
        
        \ifcoorientadordefined{
            \begin{flushright}
                \parbox{9cm}{\setlength{\parindent}{0pt}
                \textbf{Coorientador:} \getcoorientador.
                }
            \end{flushright}
        }

        \vfill

        % Data ao final da página
        {\large Boa Vista - RR\\
        \today}
    \end{center}
\end{titlepage}


% Alterar texto ou suprimir
%------------------------------------------
% Agradecimentos
% Este arquivo contém o conteúdo da página de agradecimentos do TCC.
%------------------------------------------
\begin{titlepage}
    \begin{tikzpicture}[remember picture,overlay]
        \node[opacity=0.1,inner sep=0pt] at (current page.center){
            \includegraphics[width=0.9\paperwidth]{estilo/dcc-logo.png}
        };
    \end{tikzpicture}
    \begin{center}
        % Título em negrito no topo
        {\bfseries\LARGE AGRADECIMENTOS}\\[4cm]

        
        % Texto de Agradecimentos        
        {\bfseries\Large \getagradecimento}

        

        \vfill

        % Data ao final da página
        {\large Boa Vista - RR\\
        \today}
    \end{center}
\end{titlepage}



% Sumário
\renewcommand{\contentsname}{\centering \bfseries SUMÁRIO}
\tableofcontents
\thispagestyle{empty}

\clearpage
\pagenumbering{arabic}


\maketitle


\begin{resumo} 
  \textcolor{red}{Comece com uma sentença apresentando o objetivo do estudo. Depois descreva o método. Passe para uma breve descrição de resultados, indicando, então, a conclusão do seu estudo. Finalize com uma frase que pode, por exemplo, convidar a comunidade científica para realizar mais estudos semelhantes ao seu ou que indique a contribuição do seu trabalho. O seu resumo deve ter, no máximo, 150 palavras ou 10 linhas. O mesmo vale para o abstract. Importante: em um resumo, tipicamente, não fazemos citações. Além disso, não se deve inserir fórmulas, equações, diagramas ou símbolos.}
\end{resumo}

% Resumo em inglês
\begin{abstract}
  \textcolor{red}{Dica para a produção de um resumo em inglês: traduza o texto em português com a ajuda do Google Tradutor. Depois use uma versão gratuita do Grammarly para avaliar a qualidade do texto. Aceite as sugestões desse software.}
\end{abstract}

\keywords{Word 1, Word 2, Word 3}


\section{General Information}

All full papers and posters (short papers) submitted to some SBC conference,
including any supporting documents, should be written in English or in
Portuguese. The format paper should be A4 with single column, 3.5 cm for upper
margin, 2.5 cm for bottom margin and 3.0 cm for lateral margins, without
headers or footers. The main font must be Times, 12 point nominal size, with 6
points of space before each paragraph. Page numbers must be suppressed.

Full papers must respect the page limits defined by the conference.
Conferences that publish just abstracts ask for \textbf{one}-page texts.

\section{First Page} \label{sec:firstpage}

The first page must display the paper title, the name and address of the
authors, the abstract in English and ``resumo'' in Portuguese (``resumos'' are
required only for papers written in Portuguese). The title must be centered
over the whole page, in 16 point boldface font and with 12 points of space
before itself. Author names must be centered in 12 point font, bold, all of
them disposed in the same line, separated by commas and with 12 points of
space after the title. Addresses must be centered in 12 point font, also with
12 points of space after the authors' names. E-mail addresses should be
written using font Courier New, 10 point nominal size, with 6 points of space
before and 6 points of space after.

The abstract and ``resumo'' (if is the case) must be in 12 point Times font,
indented 0.8cm on both sides. The word \textbf{Abstract} and \textbf{Resumo},
should be written in boldface and must precede the text.

\section{CD-ROMs and Printed Proceedings}

In some conferences, the papers are published on CD-ROM while only the
abstract is published in the printed Proceedings. In this case, authors are
invited to prepare two final versions of the paper. One, complete, to be
published on the CD and the other, containing only the first page, with
abstract and ``resumo'' (for papers in Portuguese).

\section{Sections and Paragraphs}

Section titles must be in boldface, 13pt, flush left. There should be an extra
12 pt of space before each title. Section numbering is optional. The first
paragraph of each section should not be indented, while the first lines of
subsequent paragraphs should be indented by 1.27 cm.

\subsection{Subsections}

The subsection titles must be in boldface, 12pt, flush left.

\section{Figures and Captions}\label{sec:figs}


Figure and table captions should be centered if less than one line
(Figure~\ref{fig:exampleFig1}), otherwise justified and indented by 0.8cm on
both margins, as shown in Figure~\ref{fig:exampleFig2}. The caption font must
be Helvetica, 10 point, boldface, with 6 points of space before and after each
caption.

\begin{figure}[ht]
\centering
\includegraphics[width=.5\textwidth]{imagens/fig1.jpg}
\caption{A typical figure}
\label{fig:exampleFig1}
\end{figure}

\begin{figure}[ht]
\centering
\includegraphics[width=.3\textwidth]{imagens/fig2.jpg}
\caption{This figure is an example of a figure caption taking more than one
  line and justified considering margins mentioned in Section~\ref{sec:figs}.}
\label{fig:exampleFig2}
\end{figure}

In tables, try to avoid the use of colored or shaded backgrounds, and avoid
thick, doubled, or unnecessary framing lines. When reporting empirical data,
do not use more decimal digits than warranted by their precision and
reproducibility. Table caption must be placed before the table (see Table 1)
and the font used must also be Helvetica, 10 point, boldface, with 6 points of
space before and after each caption.

\begin{table}[ht]
\centering
\caption{Variables to be considered on the evaluation of interaction
  techniques}
\label{tab:exTable1}
\includegraphics[width=.7\textwidth]{imagens/table.jpg}
\end{table}

\section{Images}

All images and illustrations should be in black-and-white, or gray tones,
excepting for the papers that will be electronically available (on CD-ROMs,
internet, etc.). The image resolution on paper should be about 600 dpi for
black-and-white images, and 150-300 dpi for grayscale images.  Do not include
images with excessive resolution, as they may take hours to print, without any
visible difference in the result. 

\section{References}

Bibliographic references must be unambiguous and uniform.  We recommend giving the author names references in brackets, e.g. \cite{knuth:84},
\cite{boulic:91}, and \cite{smith:99}.

The references must be listed using 12 point font size, with 6 points of space before each reference. The first line of each reference should not be indented, while the subsequent should be indented by 0.5 cm.

\bibliographystyle{estilo/sbc}
\bibliography{sbc-template}

%----------------------------
% Apêndice
% Capa personalizada para UFRR
\begin{center}        
    \vspace*{\fill}
    {\Huge \textbf{APÊNDICES}}\\
    \vspace*{\fill}
\end{center}

%------------------------------------------
% Apendice 1
%------------------------------------------
\section{APÊNDICE 01 – CRONOGRAMA DE EXECUÇÃO DA PESQUISA}
\label{sec:apendice-1}


\todo[inline]{Você pode cadastrar mais atividades. Para cada atividade, indique com um X nas colunas à direita o tempo que você irá precisar para finalizá-la.}

\begin{table}[H]
    \centering
    \begin{tabular}{|l|c|c|c|c|}
        \hline
        \multirow{2}{7.5cm}{\textbf{Atividades}} & \multicolumn{4}{c|}{\textbf{Período de execução da pesquisa}} \\ \cline{2-5}
        & \textbf{Mês 01} & \textbf{Mês 02} & \textbf{Mês 03} & \textbf{Mês 04} \\ \hline
        Atividade 01 & X & X & X & X \\ \hline
        Atividade 02 &   &   &   &   \\ \hline
        Atividade 03 &   &   &   &   \\ \hline
    \end{tabular}
\caption{Cronograma de Execução da Pesquisa}
\end{table}
    

% Anexo
% Capa personalizada para UFRR
\begin{center}        
    \vspace*{\fill}
    {\Huge \textbf{ANEXOS}}\\
    \vspace*{\fill}
\end{center}

%------------------------------------------
% Título: Anexo 01 - Termo de responsabilidade do discente
%------------------------------------------
\section{ANEXO 01 - TERMO DE RESPONSABILIDADE DO DISCENTE}
\label{sec:anexo-1}

\todo[inline]{Anexos são recursos que outro pesquisador produziu e que você usou no seu trabalho. Todos os alunos possuem, necessariamente, 03 anexos.}


\section{ANEXO 02 - DECLARAÇÃO DE AUTORIA}

\begin{center}
    \fontsize{12}{15}\selectfont
    \vspace*{0.5cm}
    \textbf{DECLARAÇÃO DE AUTORIA}
    \vspace*{1cm}
\end{center}

\vspace*{\fill}

Eu, \textbf{Nome completo do discente} (código de matricula \textbf{0000000000}), autor da monografia/TCC (Trabalho de Conclusão de Curso) sob o título \textbf{ ... }, declaro que o trabalho em referência é de minha total autoria e de minha inteira responsabilidade o texto apresentado. Declaro, ainda, que as citações e paráfrases dos  autores estão indicadas com as respectivas obras e anos de publicação. Declaro, para os devidos fins que estou ciente:
\begin{itemize}\setlength\itemsep{.02em}
    \item  dos Artigos 297 a 299 do Código Penal, Decreto-Lei n. 2.848 de 7 de dezembro de 1940;
    %
    \item  da Lei n. 9.610, de 19 de fevereiro de 1998, sobre os Direitos Autorais; e
    %
    \item  que plágio consiste na reprodução de obra alheia e submissão da mesma como trabalho próprio ou na inclusão, em trabalho próprio, de ideias, textos, tabelas ou ilustrações (quadros, figuras, gráficos, fotografias, retratos, lâminas, desenhos, organogramas, fluxogramas, plantas, mapas e outros) transcritos de obras de terceiros sem a devida e correta citação da referência.
\end{itemize}

O corpo docente responsável pela avaliação deste trabalho poderá  não  aceitar o referido trabalho caso os pontos mencionados acima sejam descumpridos, por conseguinte, considerar-me reprovado.

\vspace*{\fill}

\begin{center}
    \rule{7cm}{0.4pt} \\
    \fontsize{12}{15}\selectfont Assinatura do acadêmico(a) \\ 
    Boa Vista - RR, data (por extenso).
\end{center}

\vfill

\end{document}
