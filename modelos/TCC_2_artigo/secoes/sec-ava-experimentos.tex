%------------------------------------------
% Planejamento para Avaliação Experimental
%------------------------------------------
\section{Avaliação Experimental}
\label{sec:plan-experimentos}


\todo[inline]{\textbf{NOTA}: Deve-se adicionar um texto para introduzir o capitulo, segue um \textbf{exemplo}}

Esta seção descreve o planejamento e resultados da execução da avaliação experimental da solução proposta, incluindo: o planejamento e projeto para a execução de um estudo experimental para avaliar o método proposto 
neste trabalho.


\subsection{Projeto da Avaliação Experimental}
\label{sub:experimento}

\todo[inline]{\textbf{NOTA}: Deve-se descrever o projeto para executar os testes para avaliar a solução proposta, incluindo o ambiente, elementos que serão utilizados, artefatos de entrada e saída, dados que serão coletados, e métricas para avaliação. Segue um \textbf{exemplo} abaixo.}

\textcolor{red}{Esta seção descreve o planeamento e concepção para a execução de um estudo empírico realizado com o objetivo de avaliar a solução proposta para \textcolor{red}{XXXXX}. O estudo será conduzido aplicando a solução proposta sobre \textit{benchmarks} públicos de programas em C. Os experimentos foram conduzidos em um computador Intel Xeon CPU E5, 2.60GHz, 115GB RAM com Linux 3.13.0 - 35-generic x86\_64.}

\todo[inline]{\textbf{NOTA}: Deve-se também apresentar como será executado a avaliação e qual o seu objetivo de cada ação na avaliação. Neste momento você deve considerar as formas de testar (cenários importante) e como coletar os dados para sua avaliação. Segue um \textbf{exemplo}. } 

\textcolor{red}{Esta avaliação empírica tem como objetivo analisar a capacidade do método proposto, sobre benchmarks públicos de programas em C, para contribuir com a verificação executada pelo software $X$. Desta forma, nesta avaliação, investiga-se as seguintes questões de pesquisa (QP):
%
\begin{itemize}
    \item[QP1]: As ferramentas para a geração de invariantes são capazes de suportar as diferentes estruturas da linguagem de programação C?
    %
    \item[QP2]: As abordagens propostas para geração de dados de teste contribuem para a geração de invariantes?
    %
    \item[QP3]: As invariantes geradas contribuem na verificação executada pelo ESBMC?
\end{itemize}}


\todo[inline]{\textbf{NOTA}: Deve-se também apresentar também as etapas da experimentação.}


\subsection{Análise dos Resultados}
\label{sub:analise-resultados}

\textcolor{red}{A função desta seção é apresentar os produtos e/ou dados obtidos após a execução da solução proposta, bem como a interpretação desses dados a partir da teoria, seguida pelo apontamento das respostas às perguntas de pesquisas. Tais respostas precisam ser comparadas com o conhecimento científico existente de modo que o leitor perceba em que medida se aproximam ou se distanciam do que já era conhecido. Por fim, aponte as limitações do estudo, que são procedimentos metodológicos não realizados e que limitam a confiabilidade e a generalidade das respostas às perguntas de pesquisa.}

\textcolor{red}{Lembre-se que discutir um resultado, geralmente, significa: 1. Comparar o seu dado ou resposta com a literatura descrita na seção de Trabalhos Relacionados, apontando se o resultado foi convergente ou divergente com outros dados observados na literatura; 2. Explicar o resultado com base na literatura descrita na seção de Fundamentação Teórica, o que significa dizer se o resultado era esperado ou não. E, quando isso não for possível, levantar hipóteses que expliquem o resultado; 3. Indicar as implicações do resultado para pesquisas futuras; 4. Descrever o que significa o resultado, esclarecendo para o leitor, por exemplo, se o dado é um bom indicador ou não em relação ao que está se tentando responder na pesquisa.}

\textcolor{red}{Ao apresentar tabelas e figuras, não se limite a repetir o que já está nesses elementos gráficos. Cuide, isso sim, de destacar o que há de principal informação em uma tabela ou figura, e que o leitor não pode deixar de prestar atenção. Ou ainda, resuma o principal achado referente à tabela ou figura. Sempre adote o seguinte modelo: primeiro apresenta a tabela ou figura {“A Tabela 1 exibe…”}, depois coloque o elemento gráfico e, por fim, comente o que há de principal nele.}